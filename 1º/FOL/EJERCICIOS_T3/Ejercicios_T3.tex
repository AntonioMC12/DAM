\documentclass{article}

\usepackage{wrapfig}
\usepackage{lmodern}
\usepackage[T1]{fontenc}
\usepackage[spanish]{babel}
\usepackage{mathtools}
\usepackage{graphicx}
\usepackage[utf8]{inputenc}
\usepackage{fancyhdr}
\usepackage{enumitem}
\usepackage{hyperref}
  \hypersetup{
      colorlinks=true, %set true if you want colored links
      linktoc=all,     %set to all if you want both sections and subsections linked
      linkcolor=blue,  %choose some color if you want links to stand out
  }



\title{Ejercicios Tema 3 \\ \textbf{FOL}}
\author{Antonio Muñoz Cubero}
\date{3 de Noviembre de 2020} 
\pagestyle{fancy}

\begin{document}
  \maketitle
    \tableofcontents
      \pagenumbering{arabic}
      \lhead[FOL]{FOL}
      \lfoot[IES Francisco De Los Rios]{IES Francisco De Los Rios}
        

  \newpage
    \section{TEMA 3}
      \subsection{Ejercicio 2}
        \textbf{2º} \textit{Salva y Emilio son comerciales en una oficina de seguros situada en un antiguo bajo comercial de 45m. 
        La altura de la oficina es de 2.6m, las puertas tienen una anchura de 75cm y los pasillos de 90cm. Cuando se va la luz de 
        noche se queda todo oscuro ya que no hay ninguna luz de emergencia. \\
        \textit{¿Qué condiciones de los lugares de trabajo inclumplen 
        esta oficina?}}
        \\\\
        Las puertas deben de tener una anchura mínima de 80cm y los pasillos de 1m. Además, se debe disponer de salidas de 
        evacuación despejadas, señalizadas con ilumnación de seguridad y con puertas que se puedan abrir hacia afuera.

      \subsection{Ejercicio 3}
        \textbf{3º} \textit{Antonio trabaja en un taller de fabricación de puertas y cocinas de madera. Un día, ante la ausencia 
        del encargado, y a pesar de que no estaba autorizado, decidió utilizar la máquina de corte para cortar un tablón de madera. 
        En un momento determinado, Antonio puso en contacto su mano derecha con la cinta de corte, amputándose dos dedos. La máquina 
        de corte no disponía de marcado CE ni manual de instrucciones.}\\
        \textit{Idica qué medidas de precención/protección se debía de haber aplicado para evitar este accidente.}
        \\\\
        La maquinaría debe disponer de marcado CE como que cumple la normativa de seguridad, Dispositivos de seguridad como células 
        fotoeléctricas o el doble mando, uso de EPIs adecuados y formación en el uso de la herramienta.

        \subsection{Ejercicio 5}
        \textbf{5º} \textit{Pablo y Luis trabajan en la sección de envasado de una empresa de refrescos. Debido a la presión de la 
        empresa por aumentar la producción, han optado por inutilizar el interruptor diferencial de su sección puenteándolo para que 
        no salte constantemente. Además, en algunos lugares de la sección de envasado los cables están pelados y hay algunos empalmes 
        hechos con cinta aislante.}
        \begin{enumerate}[label=(\alph*)]
          \item \textit{¿Qué tipo de riesgos eléctricos tienen?¿Directos o indirectos?}
            \begin{itemize}
              \item Se trata de un riesgo eléctrico por contacto directo.
            \end{itemize}
          \item \textit{¿Por qué salta el diferencial?¿De qué nos informa?}
            \begin{itemize}
              \item Por el mas estado de la instalación eléctrica de la máquina, que nos indica que debe ser revisado y arreglado.
            \end{itemize}
          \item \textit{¿Qué medidas de prevención y protección debía haberse tomado?}
            \begin{itemize}
              \item No desactivar el diferencial, usar tomas de tierra correctamente, y aplicar correctamente las medidas de protección
               frente a contactos directos e indirectos.
            \end{itemize}
        \end{enumerate}

      \subsection{Ejercicio 8}
        \textbf{8º} \textit{El taller de Paco es un tanto desordenado y los trabajadores que ha contratado también se han hecho a ese
        ambiente. Lo normal es encontrar todo tipo de herramientas por el suelo, usar la primera que vienen a mano para lo que sea, 
        llevarlas de un sitio a otro a pulso a pesar de su volumen, o mantener herramientas eléctricas muy antiguas pues hasta que no 
        se rompen no se cambian.\\
        Señala de donde pueden provenir los riesgos a los que están sometidos los trabajadores en este taller y qué daños pueden ocasionar
        estos riesgos.}
        \\\\
        Están sometidos a una carga de trabajo muy elevada, que puede repercutir en su fatiga física y mental, puesto que el desorden puede 
        afectar de manera severa a la carga mental, y el continuo desplazamiento y levantamiento de herramientas contribuye a la carga física.
        Se puede solucionar ordenando todas las herramientas, optimizando el uso mental para saber donde está cada herramienta y el físico a no
        cargar de aquí para allá con pesadas herramientas de las que disponen.

      \subsection{Ejercicio 11}

        \textbf{11º} \textit{En la última medición de ruido de una empresa se detectó que el nivel de exposición de los puestos de trabajo sobrepasaba los 85db, por lo que la 
        empresa propuso una serie de medidas contra el ruido. Clasifica las siguientes medidas en función de si hacen referencia a la prevención (foco), a la protección colectiva 
        (medio de transmisión) o a la protección individual (trabajador)}
        \begin{enumerate}
          \item Prevención (foco).
          \item Protección individual (trabajador).
          \item Protección colectiva (medio de transmisión).
          \item Protección individual (trabajador).
          \item Protección colectiva (medio de transmisión).
          \item Protección individual (trabajador).
          \item Prevención (foco).
          \item Prevención (foco).
          \item Protección colectiva (medio de transmisión).
        \end{enumerate}

      \subsection{Ejercicio 13}

        \textbf{13º} \textit{Indica qué tipo de vibración (debida a transmisión mano-brazo o bien a cuerpo entero) ocasionan los siguientes equipos:}
        \begin{enumerate}[label=(\alph*)]
          \item Transmisión al cuerpo entero.
          \item Transmisión al cuerpo entero. 
          \item Transmisión mano brazo.
          \item Transmisión mano brazo.
          \item Transmisión al cuerpo entero. 
          \item Transmisión mano brazo.
          \item g) Los daños causados por la transmisión mano-brazo:
          \begin{itemize}
            \item Síndrome del dedo blanco, se pierde sensibilidad en los dedos y parte de su funcionalidad.
            \item Artrosis de codo.
            \item Lesiones de muñeca.
          \end{itemize}
        \end{enumerate}

      \subsection{Ejercicio 16}
        \textbf{13º} \textit{Leopoldo trabaja en unos invernaderos de plastico recogiendo tomates, pimientos, pepinos, entre otras verduras. En verano la temperatura llega 
        a alcanzar los 45-50ºC pues suele hacer unos 10-12ºC más que en el exterior al ser un espacio muy cerrado}
        \begin{enumerate}[label=(\alph*)]
          \item \textit{¿Qué daño principal puede causar el exceso de calor y con qué sintomas?}
            \begin{itemize}
              \item Lesiones de muñeca.Golpe de calor: provoca fiebre elevada, taquicardia, dolor de cabeza, incluso perdida de consciencia y muerte. Otros daños son como 
              lipotimias, deshidratación, mareos, calambres.
            \end{itemize}
          \item \textit{Señala 2 medidas de prevención y protección para este caso.}
            \begin{itemize}
              \item Ventilación natural y ventilación artificial. Llevar ropa adecuada y beber agua frecuentemente.
            \end{itemize}
          \item \textit{Sergio es amigo de Leopoldo, trabaja en una oficina sentado y se queja de que él en invierno siente mucho frio pues está a 18ºC ¿Está la temperatura 
          de la oficina entre los límites legales?}
            \begin{itemize}
              \item Si, es legal.
            \end{itemize}
          \item Maite, que es profesora y amiga de los dos, también se queja de que la temperatura en su clase es muy fría, pues se pone la calefacción para el turno de la mañana 
          \item y a la tarde la apagan para que se mantenga y vaya disminuyendo. Un día midió la temperatura y observó que era de 15ºC ¿Está la temperatura del aula 
          dentro de los limites legales?
            \begin{itemize}
              \item No, porque es un trabajo sedentario.
            \end{itemize}
        \end{enumerate}


      \subsection{Ejercicio 17}

        \textbf{17º} \textit{Una trabajadora baja por una escalera fija que comunica la primera planta con la planta baja de un despacho. En un momento determinado, 
        la trabajadora se resbala y cae rodadndo por cuarto escalones hasta el descansillo siguiente, lo que le produce fractura de dos costillas. 
        Tras el accidente, la empresa elabora un informe de investigación en el que establece como causas del accidente la atención inadecuada y la prisa 
        de la trabajadora, así como la falta de iluminación en la escalera, siendo ésta de uso habitual.}

        \begin{enumerate}[label=(\alph*)]
          \item \textit{¿Crees que la iluminación inadecuada puede ser causa de accidentes?¿Crees que puede existir una relación entre la intensidad 
          de la luz y el estado de ánimo de los trabajadores?}
            \begin{itemize}
              \item Si, ya que una falta de iluminación puede provocar algún tropiezo por no ver un obstáculo, yun exceso de iluminación puede cegar 
              al personal y provocar un tropiezo.Si, ya que un entorno con poca iluminación puede causar un descenso en el ánimo deltrabajador.
            \end{itemize}
          \item \textit{De las diferentes causas que indica el informe de investigación ¿cuáles serían causas técnicas y cuáles humanas?}
            \begin{itemize}
              \item La atención de la trabajadora sería un error humano y la falta de iluminación un error técnico.
            \end{itemize}
          \item \textit{Si en su puesto de trabajo las exigencias visuales fueran medias, ¿qué nivel de lux debería tener su puesto de trabajo?}
            \begin{itemize}
              \item 300 lux
            \end{itemize}
        \end{enumerate}

      \subsection{Ejercicio 22}

        \textbf{22º} \textit{Jesús, Nati y Alicia trabajan en un almacén de libros embalando y etiquetando los pedidos para que lleguen a su destino correctamente. Los libros llegan en 
        cajas desde la imprenta con 21 kg de peso y cada libro pesa 1kg. El ritmo de trabajo es frenético en temporada alta, pues los pedidos deben salir al día siguiente, embalarlos con la 
        cantidad adecuada y el libro correcto así como etiquetarlo con el cliente que toca, por lo que además de tener que llevar peso hay que fijarse mucho y llevar la información en la cabeza
        por donde van los pedidos del listado.}

        \begin{enumerate}[label=(\alph*)]
          \item \textit{¿Están los 21kg de cada caja dentro de la carga máxima general?}
            \begin{itemize}
              \item Si, el máximo es de 25kg.
            \end{itemize}
          \item \textit{Si los 3 trabajadores están perfectamente entrenados¿puden Alicia y Nati llevar esa carga de 21kg o se les aplica otro límite inferior?}
            \begin{itemize}
              \item Pueden llevar 15 kilos máximos.
            \end{itemize}
          \item \textit{¿Cual es el límite máximo para trabajadores perfectamente entrenados?}
            \begin{itemize}
              \item 40kg
            \end{itemize}
          \item \textit{Señala las 4 indicaciones básicas para manipular cargas manualmente de forma correcta}
            \begin{itemize}
              \item Apoyar firmes los pies en el suelo y separarlos. \\Doblar las rodillas para agarrar la carga (nunca doblar la espalda). \\Sujetar firme y cargarla cerca del cuerpo. \\Mantener la espalda recta y no curvada.
            \end{itemize}
          \item \textit{¿Crees que los trabjadores están sometidos a carga mental?}
            \begin{itemize}
              \item Si, por estar en temporada alta, donde hacen mucho en muy poco tiempo.
            \end{itemize}
          \item \textit{¿Señala una medida de prevención y protección para la carga física y otra para la carga mental?}
            \begin{itemize}
              \item Sustituir la manipulación manual de carga por máquinas y Adaptar la cantidad de informacióna la capacidad del trabajador.
            \end{itemize}
        \end{enumerate}
      
      \subsection{Ejercicio 23}

        \textbf{22º} \textit{Indica si las siguientes situaciones se deben a Insatisfacción laboral, estrés, burnout o mobbing e indica por qué:}
        
        \begin{enumerate}[label=(\alph*)]
          \item \textit{Me han trasladado a un despacho sin compañeros y sin tener nada que hacer.}
            \begin{itemize}
              \item Mobbing, ya que es acoso psicológico hacia el trabajador.
            \end{itemize}
          \item \textit{No tengo la ilusión de antes, no puedo más, necesito vacaciones.}
            \begin{itemize}
              \item Burnout, porque el trabajador está harto del trabajo.
            \end{itemize}
          \item \textit{Como no vuelva pronto la administrativa de baja no voy a poder procesar toda la documentación pendiente de este mes, ya que me supera}
            \begin{itemize}
              \item Estrés, porque el trabajador está abrumado por la cantidad de trabajo.
            \end{itemize}
          \item \textit{Yo me esperaba ganar 1000€ porque es lo que me dijeron en la entrevista, pero resulta que con los descuentos se me queda en 850€, me han engañado.}
            \begin{itemize}
              \item Insatisfacción laboral, porque no está contento con su trabajo.
            \end{itemize}
          \item \textit{Cuando hablo los compañeros me ignoran como si no hubiese nadie.}
            \begin{itemize}
              \item Mobbing, porque los compañeros están haciendo un ataque contra su bienestar psicológico.
            \end{itemize}
        \end{enumerate}
\end{document}