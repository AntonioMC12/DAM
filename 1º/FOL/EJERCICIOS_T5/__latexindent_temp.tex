\documentclass{article}

\usepackage{wrapfig}
\usepackage{lmodern}
\usepackage[T1]{fontenc}
\usepackage[spanish]{babel}
\usepackage{mathtools}
\usepackage{graphicx}
\usepackage[utf8]{inputenc}
\usepackage{fancyhdr}
\usepackage{enumitem}
\usepackage{hyperref}
  \hypersetup{
      colorlinks=true, %set true if you want colored links
      linktoc=all,     %set to all if you want both sections and subsections linked
      linkcolor=blue,  %choose some color if you want links to stand out
  }



\title{Ejercicios Tema 3 \\ \textbf{FOL}}
\author{Antonio Muñoz Cubero}
\date{22 de Noviembre de 2020} 
\pagestyle{fancy}

\begin{document}
  \maketitle
    \tableofcontents
      \pagenumbering{arabic}
      \lhead[FOL]{FOL}
      \lfoot[IES Francisco De Los Rios]{IES Francisco De Los Rios}
        

  \newpage
    \section{TEMA 4}
      \subsection{Ejercicio 1}
        \begin{enumerate}[label=(\alph*)]
          \item \textit{Utilizar los ascensores y salir corriendo lo máximo posible:}
            \begin{itemize}
              \item No son adecuados. Porque si el hueco del ascensor se llena de humo, en caso de incendio el hueco del ascensor funciona como chimenea, y no correr para que no cunda el pánico y salir mas ordenadamente sin dejar a nadie atrás.
            \end{itemize}
          \item \textit{Volver a los despachos a por los abrigos y a cerrar con llave:}
            \begin{itemize}
              \item No es adecuado. Porque es peligroso y puedes quedarte atrapado.
            \end{itemize}
          \item \textit{Si hay humo cubrirse la nariz y la boca con un pañuelo mojado y andar a gatas:}
            \begin{itemize}
              \item Si es adecuado. Porque se puede respirar mejor en caso de humo a través de un trapo mojado y a gatas porque el aire caliente sube.
            \end{itemize}
          \item \textit{Dirigirse al punto de reunión de forma ordenada según las instrucciones de los equipos de alarma y evacuación:}
            \begin{itemize}
              \item Si es adecuado. Porque así se tiene un orden y se puede ver si falta alguien.
            \end{itemize}
          \item \textit{Si se prende fuego e nla ropa no correr sino rodar por el suelo:}
            \begin{itemize}
              \item Si es adecuado. Porque puedes apagar el fuego de tu ropa rodando.
            \end{itemize}
          \item \textit{Volver a por el móvil y dejar las puertas abiertas:}
            \begin{itemize}
              \item No es adecuado. Porque es peligroso volver al edificio y además hay que dejar las puertas cerradas para intentar contener el fuego.
            \end{itemize}
        \end{enumerate}
\end{document}