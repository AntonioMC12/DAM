\documentclass{article}

\usepackage{wrapfig}
\usepackage{lmodern}
\usepackage[T1]{fontenc}
\usepackage{graphicx}
\usepackage[utf8]{inputenc}


\title{TEMA 5 BASES DE DATOS \\ El Modelo Relacional}
\author{Antonio Muñoz Cubero}
\date{5 de Ocutbre de 2020} 

\begin{document}
\maketitle
\pagenumbering{gobble}

\newpage
\tableofcontents
\pagenumbering{roman}

\newpage
\section{Introducción}
Propone una representación de la información en forma de tablas bidimensionales, una tabla es una matriz rectangular identificada
por un nombre único en todo el modelo y en la cual cada elemento de la tabla es un 'item' de datos particular.
\\
Con esta forma de representación intentaremos implantar una base de datos en un sistema informático concreto
\\
\\
Características del Modelo:

  \begin{itemize}
    \item \textbf{Idenpendencia física}: \textit{Idependencia fisica en las bases de datos relacionales de la forma en la que se almacenan los datos es completamente independiente
    de su manipulación lógica.}
    \item \textbf{Idenpendencia lógica}: \textit{La modificación de cualquiera de los elementos de la base de datos en principio no influirá en los
    programas que la usen.}
    \item \textbf{Flixibilidad}: \textit{Permitecer los datos de la forma más adecuada a nuestras necesidades.}
    \item \textbf{Uniformidad}: \textit{toda la información se guarda de manera uniforme, en forma de tablas, lo que facilita su manipulación.}
  \end{itemize}

\section{Elementos Del Modelo Relacional}
En este modelo se guardarán las tablas como componentes principales, también llamadas relaciones, y junto a ellas las interrelaciones entre 
las distintas tablas, las restricciones y los procedimientos.

\subsection{Relaciones o Tablas}
Es el elemento principal y básico del modelo relacional, y será una \textbf{relación o lista de valores}. Cada relación tendrá su \textbf{nombre}, que será único 
en todo el modelo, \textbf{atributos}, que son las propiedades de la tabla y forman las \textbf{columnas}. Dan significado a los datos de esa columna y \underline{su nombres
será único dentro de caada tabla} y por último, \textbf{tuplas} o filas de la tabla, son las ocurrencias, son los datos en sí mismo. 
\\
A partir de eso, sacamos otros 3 conceptos:
\\
  \begin{itemize}
    \item \textbf{Grado}: \textit{Es su númro de atributos.}
    \item \textbf{Cardinalidad}: \textit{Número de tuplas de una tabla.}
    \item \textbf{Valor}: \textit{Es el cruce de una fila y una columna. es un item concreto de datos. Puede haber valores nulos y vacíos..}
  \end{itemize}

\newpage
Una tabla por tanto se compone de una cabecera que define su estructura, que es fija y un cuerpo que varía a lo largo del tiempo.
\\
\subsubsection{Representación de las Tablas}
Para representar las tablas, se puede hacer de dos formas:
\\
  \begin{itemize}
    \item \textbf{De forma Textual}: \textit{En esta representación las tablas se indican primero por su nombre y a continuación entre paréntesis los atributos que tiene 
    La clave primaria se subraya con un trazo continuo y las claves alternativas (si las hubiera) con trazo discontinuo. Esta será la representación
    que utilicemos nosotros.}
    \item \textbf{De forma Gráfica}:
  \end{itemize}
\subsubsection{Claves}
En todas las Relaciones o Tablas, debe haber \textbf{un atributo o conjunto de estos que tome valores únicos para cada una de las tuplas} de la relación. Estas
serán las claves candidatas. De entre todas las que haya, se elegirá una como \textbf{clave principal y el resto serán claves alternativas}. A la hora de escoger
la clave primaria, en caso de que haya mas de una opción, deberíamos elegir la que más se vaya a utilizar en el problema.
\end{document}
