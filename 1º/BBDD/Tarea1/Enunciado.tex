\documentclass{article}

\usepackage{wrapfig}
\usepackage{lmodern}
\usepackage[T1]{fontenc}
\usepackage[spanish]{babel}
\usepackage{mathtools}
\usepackage{graphicx}
\usepackage[utf8]{inputenc}
\usepackage{fancyhdr}



\title{Documentación \\\large \textbf{Problema Base de Datos}}
\author{Antonio Muñoz Cubero y Miguel Ángel García Mérida}
\date{20 de Ocutbre de 2020} 


\begin{document}
\maketitle
\pagenumbering{gobble}

\newpage
\section{Enunciado}
Se desea mantener información sobre las ventad de paquetes de tabaco de las diferentes expendedoras autorizadas. Es importante para el distribuidor conocer información
sobre las expendedoras, los pedidos que realizan y las ventas de las mismas. Es importante conocer datos sobre las ventallas detalladas para cada una de las clases de 
tabaco.
\\
Los procesos de consulta mas usuales que se realizan harán referencia a la venta y distribución de tabaco y el consumo que se realiza , bajo cualquier tipo de agrupación,
de los tabacos vendidos por las expendedoras.
\\
\\
\textbf{Aclaraciones:}
\\
\begin{enumerate}
  
  \item Los estancos son abastecidos con un número diferente de cigarros que solo depende de la orden del pedido y del numero de existencias de dicho tipo. Por lo tanto, los 
  estancos podrán vender cualquier tipo de cigarrillo de los que tengan existencias.
  
  \item Los estancos tienen asignado un número de expendedoras que se pueden repetir de una localidad a otra. Además, cada estanco tiene asignado un número de identificación 
  fiscal que corresponde a la empresa como tal, así como un nombre, el cual puede ser el del responsable o no, que puede repetirse incluse en la misma localidad.
  
  \item Los fabricantes de cigarrillos tiene una sede central en un pais, aunque en un mismo pais se pueden encontrar sedes de varios fabricantes.
  
  \item Cada fabricante puede fabricar un número variable de marcas de cigarrillos, si bien, una marca de cigarrillos, independientemente de su tipo, solo puede ser fabricada
  por un único fabricante.

  \item Para cada marca de cigarrillos se fabrican distintos tipos de ellos, según la existencia o no de filtro, el color de la hoja de tabaco y la cantidad de nicotina y 
  alquitran existente en los mismos.

  \item De una misma marca pueden existir cigarrillos con y sin filtro.
  
  \item De una misma marca pueden existir cigarrillos con hoja blanca y negra. 
  
  \item De una misma marca pueden existir cigarrillos con contenido de contaminantes (nicotina y alquitran) catalogados, según el fabricante, en: \textit{Light, SuperLight y UltraLight},
  cuando un tipo no está catalogado en uno de estos grupos (no ha tenido tratamiento especial para eliminar parte de los componentes), se considera que es \textit{Normal}. Los cigarrillos
  sin filtro serán catalogados como \textit{Normal}.

  \item De una misma marca pueden existir cigarrillo mentolados y no mentolados. Los cigarrillos mentolados tienen siempre un contenido de contaminantes \textit{Normal}.
  
  \item No me interesa conocer los clientes a los que se venden los cigarrillos en los estancos
  
  \item Tanto para las compras como para las ventas, sólo interesa conocer el total de ellas, de cada tipo de cigarrillos, realizadas por día.  

\end{enumerate}

\end{document}