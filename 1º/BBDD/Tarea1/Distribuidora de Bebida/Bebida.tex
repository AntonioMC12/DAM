\documentclass{article}

\usepackage{wrapfig}
\usepackage{lmodern}
\usepackage[T1]{fontenc}
\usepackage[spanish]{babel}
\usepackage{mathtools}
\usepackage{graphicx}
\usepackage[utf8]{inputenc}
\usepackage{fancyhdr}



\title{Documentación \\\large \textbf{Problema Base de Datos}}
\author{Antonio Muñoz Cubero y Miguel Ángel García Mérida}
\date{22 de Ocutbre de 2020} 


\begin{document}
  \maketitle
  \pagenumbering{gobble}
  \newpage
    \section{Enunciado}
      El alcalde de una localidad nos ha contactado para que realicemos la base de datos de los clubes deportivos que hay en la localidad, puesto que necesita llevar un 
      control de estos y sus socios para mejorar en gran medida la organización de los torneos y competiciones, ya que tienen que tener regristrado todos los datos de estos. 
      Nos pide que por favor, atendamos a los siguientes supuestos:
    \\
    \\
    \textbf{Supuestos:}
    \\
      \begin{enumerate}
        
        \item Un club puede usar varias instalaciones y estas puden ser usadas por varios clubs.
        \item Las instalaciones tienen un id que las identifican, una ubicación y un deporte que para las que se usan.
        \item Un club tiene un id que lo identifica, un presidente, un nombre y un nombre oficial, una fecha de fundación, un teléfono y un correo electrónico.
        \item Los socios solo pueden pertenecer a un único club.
        \item Los socios tienen un id que los identifican.

      \end{enumerate}

\end{document}