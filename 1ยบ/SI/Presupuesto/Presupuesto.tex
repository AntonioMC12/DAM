\documentclass{article}

\usepackage[utf8]{inputenc}
\usepackage[T1]{fontenc}
\usepackage[spanish]{babel}
\usepackage{times}
\usepackage{wrapfig}
\usepackage{lmodern}
\usepackage{mathtools}
\usepackage{graphicx}
\usepackage[utf8]{inputenc}
\usepackage{color}
\usepackage{hyperref}
\usepackage{fancyhdr,lipsum}


\hypersetup{
    colorlinks=true, %set true if you want colored links
    linktoc=all,     %set to all if you want both sections and subsections linked
    linkcolor=blue,  %choose some color if you want links to stand out
}
\urlstyle{same}

\title{Presupuesto}
\author{Antonio Muñoz Cubero}
\date{3 de Noviembre de 2020} 

\begin{document}
  \maketitle
  \pagenumbering{gobble}
  \pagestyle{fancy}
  
  \newpage
    \tableofcontents
    \lhead[Presupuesto SSII]{Presupuesto SSII}
    \lfoot[IES Francisco De Los Rios]{IES Francisco De Los Rios}
  
  \newpage
    \pagenumbering{roman}

  \newpage
    \section{Enunciado}
      Utilizando como base el trabajo realizado en el ejercicio anterior, realizar un diseño de equipo de escritorio empleando componentes reales que se encuentren en el mercado. Se puede emplear como lista de materiales y precios cualquier tienda real física o de Internet (PC Componentes, PC Box, etc.). Justificar en todo momento la compatibilidad de los componentes empleados con la placa base escogida.
      \begin{itemize}
        \item Realizar un dimensionamiento correcto de la potencia necesaria para la fuente de alimentación.
        \item Realizar el calculo del coste de material empleando una hoja de calculo (excel o similar).
      \end{itemize}
      Se debe entregar el documento en formato PDF, Entregar también el presupuesto elaborado con la hoja de calculo (puede estar también impreso en formato PDF).
      \\\\
      \subsection{Aclaraciones}
      Este trabajo es una continuación del anterior, que en mi caso fue la \href{https://github.com/ErTonix12/DAM/blob/main/1%C2%BA/SI/Placa_Base_Documentaci%C3%B3n/build/Placa_Base_Documentacion.pdf}{documentación de la placa base}, que puede ver pulsando en el enlace. (No puede ser visualizado online en el navegador \textit{Microsoft EDGE}.)
      \\
      En este presupuesto utilizaré la placa base anterior, y construiré un \textit{PC} en base a ella, siendo este presupuesto válido en caso de hacerlo efectivo, pues las compatibilidad de todos los componentes están más que testados.
  \newpage
      \section{Procesador}
        \subsection{Descripción}
          Para el apartado del \textit{procesador} he elegido uno de los procesadores más potentes del mercado y uno de los más potentes dentro del fabricante \textit{AMD}, se trada de \textbf{AMD Ryzen 9 3900X}.
          \\
          Este procesador pertenece a la tercena generación de procesadores AMD Ryzen

          

\end{document}

