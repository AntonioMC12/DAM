\documentclass{article}

\usepackage{wrapfig}
\usepackage{lmodern}
\usepackage[T1]{fontenc}
\usepackage[spanish]{babel}
\usepackage{mathtools}
\usepackage{graphicx}
\usepackage[utf8]{inputenc}
\usepackage[a4paper]{geometry}
\usepackage{fancyhdr}
\usepackage{enumitem}
\usepackage{hyperref}
  \hypersetup{
      colorlinks=true, %set true if you want colored links
      linktoc=all,     %set to all if you want both sections and subsections linked
      linkcolor=blue,  %choose some color if you want links to stand out
  }
  \geometry{top=2.5cm, bottom=2.5cm}



\title{Ejercicios Tema 5 \\ \textbf{FOL}}
\author{Antonio Muñoz Cubero}

\pagestyle{fancy}

\begin{document}
  \maketitle
    \tableofcontents
      \pagenumbering{arabic}
      \lhead[FOL]{FOL}
      \lfoot[IES Francisco De Los Rios]{IES Francisco De Los Rios}
        

  \newpage
    \section{TEMA 5}
      \subsection{Ejercicio 1}
        \begin{enumerate}[label=(\alph*)]
          \item \textit{¿Cuál es la relación del vendedor de libros?}
            \begin{itemize}
              \item Es un agente comercial, no tiene relación laboral
            \end{itemize}
          \item \textit{La voluntaria de la ONG, ¿Qué requisito no cumple para no ser laboral?}
            \begin{itemize}
              \item Realiza un trabajo de buena vecindad o benevolencia, por lo tanto no es una relación laboral.
            \end{itemize}
          \item \textit{¿Tiene Laura derecho a reclamar que su relación fue laboral?¿Por qué?}
            \begin{itemize}
              \item Las prestaciones personales que sean obligatorias de hacer, están excluidas de relación laboral.
            \end{itemize}
        \end{enumerate}

  \newpage
      \subsection{Ejercicio 8}
        \begin{enumerate}[label=(\alph*)]
          \item \textit{¿Es legal que gane más que el convenio?¿Por qué?}
            \begin{itemize}
              \item Si es legal, ya que se trata de una condición más beneficiosa del contrato respecto a las normal superiores.
            \end{itemize}
          \item \textit{¿Es legal que el convenio mejore el SMI de 900€?¿Por qué?}
            \begin{itemize}
              \item Sería legal, mientras la condición sea mas beneficiosa que las leyes superiores.
            \end{itemize}
          \item \textit{¿Puede el trabajador pactar en el contrato el ganar por debajo del convenio?¿Por qué?}
            \begin{itemize}
              \item No se puede empeorar las condiciones del convenio colectivo, sí se podría mejorar, pero nunca empeorar.
            \end{itemize}
        \end{enumerate}

  \newpage
      \subsection{Ejercicio 13}
        \begin{enumerate}[label=(\alph*)]
          \item \textit{¿Puede Mari Carmen negarse a cumplir esa orden?¿Por qué motivo?}
            \begin{itemize}
              \item Puede negarse a cumplir dicha orden, puesto que pone en riesgo al salud del trabajador y en los deberes del trabajador está cláramente descrito que 
              debe tomar las medidas de seguridad necesarias y puede negarse a las órdenes del empresario siempre que supongan un daó para el trabajador.
            \end{itemize}
          \item \textit{¿Qué debe hacer como norma general Diana ante la instrucción recibida?}
            \begin{itemize}
              \item Como norma general, primero cumpliría la orden y luego se reclama ante el juzgado si se está en desacuerdo. 
            \end{itemize}
          \item \textit{¿En qué casos puede Diana resistirse a cumplirla?}
            \begin{itemize}
              \item Los casos de resistencia se dan en las órdenes que ponen en peligro la salud del trabajador, que afecten a su vida privada o atenten contra 
              su dignidad.
            \end{itemize}
        \end{enumerate}
\end{document}