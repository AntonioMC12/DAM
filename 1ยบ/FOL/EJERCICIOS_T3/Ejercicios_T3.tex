\documentclass{article}

\usepackage{wrapfig}
\usepackage{lmodern}
\usepackage[T1]{fontenc}
\usepackage[spanish]{babel}
\usepackage{mathtools}
\usepackage{graphicx}
\usepackage[utf8]{inputenc}
\usepackage{fancyhdr}
\usepackage{enumitem}
\usepackage{hyperref}
  \hypersetup{
      colorlinks=true, %set true if you want colored links
      linktoc=all,     %set to all if you want both sections and subsections linked
      linkcolor=blue,  %choose some color if you want links to stand out
  }



\title{Ejercicios Tema 3 \\ \textbf{FOL}}
\author{Antonio Muñoz Cubero}
\date{3 de Noviembre de 2020} 
\pagestyle{fancy}

\begin{document}
  \maketitle
    \tableofcontents
      \pagenumbering{arabic}
      \lhead[FOL]{FOL}
      \lfoot[IES Francisco De Los Rios]{IES Francisco De Los Rios}
        

  \newpage
    \section{TEMA 3}
      \subsection{Ejercicio 2}
        \textbf{2º} \textit{Salva y Emilio son comerciales en una oficina de seguros situada en un antiguo bajo comercial de 45m. 
        La altura de la oficina es de 2.6m, las puertas tienen una anchura de 75cm y los pasillos de 90cm. Cuando se va la luz de 
        noche se queda todo oscuro ya que no hay ninguna luz de emergencia. \\
        \textit{¿Qué condiciones de los lugares de trabajo inclumplen 
        esta oficina?}}
        \\\\
        Las puertas deben de tener una anchura mínima de 80cm y los pasillos de 1m. Además, se debe disponer de salidas de 
        evacuación despejadas, señalizadas con ilumnación de seguridad y con puertas que se puedan abrir hacia afuera.

        \subsection{Ejercicio 3}
        \textbf{3º} \textit{Antonio trabaja en un taller de fabricación de puertas y cocinas de madera. Un día, ante la ausencia 
        del encargado, y a pesar de que no estaba autorizado, decidió utilizar la máquina de corte para cortar un tablón de madera. 
        En un momento determinado, Antonio puso en contacto su mano derecha con la cinta de corte, amputándose dos dedos. La máquina 
        de corte no disponía de marcado CE ni manual de instrucciones.}\\
        \textit{Idica qué medidas de precención/protección se debía de haber aplicado para evitar este accidente.}
        \\\\
        La maquinaría debe disponer de marcado CE como que cumple la normativa de seguridad, Dispositivos de seguridad como células 
        fotoeléctricas o el doble mando, uso de EPIs adecuados y formación en el uso de la herramienta.

        \subsection{Ejercicio 5}
        \textbf{5º} \textit{Pablo y Luis trabajan en la sección de envasado de una empresa de refrescos. Debido a la presión de la 
        empresa por aumentar la producción, han optado por inutilizar el interruptor diferencial de su sección puenteándolo para que 
        no salte constantemente. Además, en algunos lugares de la sección de envasado los cables están pelados y hay algunos empalmes 
        hechos con cinta aislante.}
        \begin{enumerate}[label=(\alph*)]
          \item \textit{¿Qué tipo de riesgos eléctricos tienen?¿Directos o indirectos?}
            \begin{itemize}
              \item Se trata de un riesgo eléctrico por contacto directo.
            \end{itemize}
          \item \textit{¿Por qué salta el diferencial?¿De qué nos informa?}
            \begin{itemize}
              \item Por el mas estado de la instalación eléctrica de la máquina, que nos indica que debe ser revisado y arreglado.
            \end{itemize}
          \item \textit{¿Qué medidas de prevención y protección debía haberse tomado?}
            \begin{itemize}
              \item No desactivar el diferencial, usar tomas de tierra correctamente, y aplicar correctamente las medidas de protección
               frente a contactos directos e indirectos.
            \end{itemize}
        \end{enumerate}

        \subsection{Ejercicio 8}
        \textbf{8º} \textit{El taller de Paco es un tanto desordenado y los trabajadores que ha contratado también se han hecho a ese
        ambiente. Lo normal es encontrar todo tipo de herramientas por el suelo, usar la primera que vienen a mano para lo que sea, 
        llevarlas de un sitio a otro a pulso a pesar de su volumen, o mantener herramientas eléctricas muy antiguas pues hasta que no 
        se rompen no se cambian.\\
        Señala de donde pueden provenir los riesgos a los que están sometidos los trabajadores en este taller y qué daños pueden ocasionar
        estos riesgos.}
        \\\\
        Están sometidos a una carga de trabajo muy elevada, que puede repercutir en su fatiga física y mental, puesto que el desorden puede 
        afectar de manera severa a la carga mental, y el continuo desplazamiento y levantamiento de herramientas contribuye a la carga física.
        Se puede solucionar ordenando todas las herramientas, optimizando el uso mental para saber donde está cada herramienta y el físico a no
        cargar de aquí para allá con pesadas herramientas de las que disponen.


\end{document}