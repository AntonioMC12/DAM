\documentclass{article}

\usepackage{wrapfig}
\usepackage{lmodern}
\usepackage[T1]{fontenc}
\usepackage[spanish]{babel}
\usepackage{mathtools}
\usepackage{graphicx}
\usepackage[utf8]{inputenc}
\usepackage{fancyhdr}
\usepackage{enumitem}



\title{Ejercicios Tema 1 y 2 FOL}
\author{Antonio Muñoz Cubero}
\date{3 de Noviembre de 2020} 


\begin{document}
  \maketitle
    \tableofcontents
      \pagenumbering{gobble}

  \newpage
    \section{TEMA 1}
      \subsection{Ejercicio 1}
        \textbf{1º} \textit{Indica para el siguiente trabajador de un almacén de cartonaje a qué factores de riesgo está sometido, así como dentro de cada grupo el tipo de factor de riesgo que es (lugares, equipos,físicos,etc).}
        \\
        \begin{enumerate}[label=(\alph*)]
          \item \textit{Debe transportar los palés de cartón en la zenwich (maquinaria de transporte de palés).}
            \begin{itemize}
              \item Está carente de condiciones de seguridad con los equipos de trabajo.
            \end{itemize}
          \item \textit{Mover cajas de 25kg para colocarlas correctamente en las estanterias.}
            \begin{itemize}
              \item Está carente de condiciones ergonómicas por la carga física.
            \end{itemize}
          \item \textit{Tiene un ruido de fondo de 85 decibelios.}
            \begin{itemize}
              \item Está carente de condiciones medioambientales debido a los agentes físicos.
            \end{itemize}
          \item \textit{Utiliza una serradora de cartón para cortar el cartón sobrante en trozos mas pequeños.}
            \begin{itemize}
              \item Está carente de condiciones de seguridad por el equipo de trabajo que usa.
            \end{itemize}
          \item \textit{En verano en el almacén se está al menos a 30ºC.}
            \begin{itemize}
              \item Está carente de condiciones medioambientales debido a los agentes físicos.
            \end{itemize}
          \item \textit{Su trabajo se le hace monótono y aburrido.}
            \begin{itemize}
              \item Está carente de condiciones psicosociales debido a la organización del trabajo.
            \end{itemize}
          \item \textit{Maneja mucha información de los pedidos y las cajas.}
            \begin{itemize}
              \item Está carente de condiciones ergonómicas debido a la carga mental de su trabajo.
            \end{itemize}
          \item \textit{El polvo que levanta el cartón le provoca estornudos y alergias.}
            \begin{itemize}
              \item Está carente de condiciones medioambientales por agentes químicos.
            \end{itemize}
          \item \textit{Pero él es un perfeccionista y podrá con todo sin importarle los riesgos que existan.}
            \begin{itemize}
              \item Está carente de condiciones psicosociales por las características personales.
            \end{itemize}
        \end{enumerate}
    \newpage
      \subsection{Ejercicio 2}
        \textbf{2º} \textit{En una empresa se han dado distintos accidentes a lo largo del año. Los trabajadores quisieran saber si desde el 
        punto de vista legal se considera o no accidente de trabajo para estar de baja laboral. Indica si es accidente de trabajo desde el 
        punto de vista de la Seguridad Social y porqué.}
        \\
        \begin{enumerate}[label=(\alph*)]
          \item \textit{una trabajadora sufre un corte en un dedo al manipular una máquina en el trabajo.}
            \begin{itemize}
              \item 
            \end{itemize}
          \item \textit{Un golpe con el coche al dirigirse al trabajo que obliga a ponerse el collarín y reposo}
            \begin{itemize}
              \item 
            \end{itemize}
          \item \textit{El mismo accidente de coche anterior pero ocurrido un domingo por la noche cuando se dirige al domicilio laboral para empezar a trabajar el lunes.}
            \begin{itemize}
              \item 
            \end{itemize}
          \item \textit{Al conserje le indican que acuda al almacén a ayudar en la descarga de un pedido, sufriendo un golpe por la caída de una caja voluminosa.}
            \begin{itemize}
              \item 
            \end{itemize}
          \item \textit{Una depresión producida por acoso laboral, la cual se demuestra en el juicio que está vinculada al trabajo.}
            \begin{itemize}
              \item 
            \end{itemize}
          \item \textit{}
            \begin{itemize}
              \item
            \end{itemize}
          \item \textit{}
            \begin{itemize}
              \item 
            \end{itemize}
          \item \textit{}
            \begin{itemize}
              \item 
            \end{itemize}
          \item \textit{}
            \begin{itemize}
              \item 
            \end{itemize}
        \end{enumerate}
    
    \newpage
      \subsection{Ejercicio 7}
        Enunciado
        \\
        \begin{enumerate}[label=(\alph*)]
          \item \textit{}
            \begin{itemize}
              \item 
            \end{itemize}
          \item \textit{}
            \begin{itemize}
              \item 
            \end{itemize}
          \item \textit{}
            \begin{itemize}
              \item 
            \end{itemize}
          \item \textit{}
            \begin{itemize}
              \item 
            \end{itemize}
          \item \textit{}
            \begin{itemize}
              \item 
            \end{itemize}
          \item \textit{}
            \begin{itemize}
              \item
            \end{itemize}
          \item \textit{}
            \begin{itemize}
              \item 
            \end{itemize}
          \item \textit{}
            \begin{itemize}
              \item 
            \end{itemize}
          \item \textit{}
            \begin{itemize}
              \item 
            \end{itemize}
        \end{enumerate}



\end{document}
