\documentclass{article}

\usepackage{wrapfig}
\usepackage{lmodern}
\usepackage[T1]{fontenc}
\usepackage[spanish]{babel}
\usepackage{mathtools}
\usepackage{graphicx}
\usepackage[utf8]{inputenc}
\usepackage{fancyhdr}
\usepackage{enumitem}
\usepackage{hyperref}
  \hypersetup{
      colorlinks=true, %set true if you want colored links
      linktoc=all,     %set to all if you want both sections and subsections linked
      linkcolor=blue,  %choose some color if you want links to stand out
  }



\title{Ejercicios Tema 1 y 2 \\ \textbf{FOL}}
\author{Antonio Muñoz Cubero}
\date{3 de Noviembre de 2020} 


\begin{document}
  \maketitle
    \tableofcontents
      \pagenumbering{gobble}

  \newpage
    \section{TEMA 1}
      \subsection{Ejercicio 1}
        \textbf{1º} \textit{Indica para el siguiente trabajador de un almacén de cartonaje a qué factores de riesgo está sometido, así como dentro de cada grupo el tipo de factor de riesgo que es (lugares, equipos,físicos,etc).}
        \\
        \begin{enumerate}[label=(\alph*)]
          \item \textit{Debe transportar los palés de cartón en la zenwich (maquinaria de transporte de palés).}
            \begin{itemize}
              \item Está carente de condiciones de seguridad con los equipos de trabajo.
            \end{itemize}
          \item \textit{Mover cajas de 25kg para colocarlas correctamente en las estanterias.}
            \begin{itemize}
              \item Está carente de condiciones ergonómicas por la carga física.
            \end{itemize}
          \item \textit{Tiene un ruido de fondo de 85 decibelios.}
            \begin{itemize}
              \item Está carente de condiciones medioambientales debido a los agentes físicos.
            \end{itemize}
          \item \textit{Utiliza una serradora de cartón para cortar el cartón sobrante en trozos mas pequeños.}
            \begin{itemize}
              \item Está carente de condiciones de seguridad por el equipo de trabajo que usa.
            \end{itemize}
          \item \textit{En verano en el almacén se está al menos a 30ºC.}
            \begin{itemize}
              \item Está carente de condiciones medioambientales debido a los agentes físicos.
            \end{itemize}
          \item \textit{Su trabajo se le hace monótono y aburrido.}
            \begin{itemize}
              \item Está carente de condiciones psicosociales debido a la organización del trabajo.
            \end{itemize}
          \item \textit{Maneja mucha información de los pedidos y las cajas.}
            \begin{itemize}
              \item Está carente de condiciones ergonómicas debido a la carga mental de su trabajo.
            \end{itemize}
          \item \textit{El polvo que levanta el cartón le provoca estornudos y alergias.}
            \begin{itemize}
              \item Está carente de condiciones medioambientales por agentes químicos.
            \end{itemize}
          \item \textit{Pero él es un perfeccionista y podrá con todo sin importarle los riesgos que existan.}
            \begin{itemize}
              \item Está carente de condiciones psicosociales por las características personales.
            \end{itemize}
        \end{enumerate}
    \newpage
      \subsection{Ejercicio 2}
        \textbf{2º} \textit{En una empresa se han dado distintos accidentes a lo largo del año. Los trabajadores quisieran saber si desde el 
        punto de vista legal se considera o no accidente de trabajo para estar de baja laboral. Indica si es accidente de trabajo desde el 
        punto de vista de la Seguridad Social y porqué.}
        \\
        \begin{enumerate}[label=(\alph*)]
          \item \textit{una trabajadora sufre un corte en un dedo al manipular una máquina en el trabajo.}
            \begin{itemize}
              \item 
            \end{itemize}
          \item \textit{Un golpe con el coche al dirigirse al trabajo que obliga a ponerse el collarín y reposo}
            \begin{itemize}
              \item 
            \end{itemize}
          \item \textit{El mismo accidente de coche anterior pero ocurrido un domingo por la noche cuando se dirige al domicilio laboral para empezar a trabajar el lunes.}
            \begin{itemize}
              \item 
            \end{itemize}
          \item \textit{Al conserje le indican que acuda al almacén a ayudar en la descarga de un pedido, sufriendo un golpe por la caída de una caja voluminosa.}
            \begin{itemize}
              \item 
            \end{itemize}
          \item \textit{Una depresión producida por acoso laboral, la cual se demuestra en el juicio que está vinculada al trabajo.}
            \begin{itemize}
              \item 
            \end{itemize}
          \item \textit{Mientras manipulaba el curvado de acero, un trabajador se distrae hablando con un compañero y se amputa un dedo}
            \begin{itemize}
              \item
            \end{itemize}
          \item \textit{En una empresa de Alzira se produce una inundación un lunes por la mañana por las lluvias torrenciales, fruto de lo cual un trabjador resbala golpeándose gravemente la cadera.}
            \begin{itemize}
              \item 
            \end{itemize}
          \item \textit{La directora financiera acude al banco a firmar unos pagarés y durante el camino es atropellada por una moto.}
            \begin{itemize}
              \item 
            \end{itemize}
          \item \textit{Un albañil experimentado sufre una caída con consecuencias graves, pues decide que para 4 metros de altura, no se pone arnés.}
            \begin{itemize}
              \item 
            \end{itemize}
        \end{enumerate}
    
    \newpage
      \subsection{Ejercicio 7}
        \textbf{7º} \textit{Clasifica las siguientes medidas según sean medidas de prevención o de ptrotección. Indica sobre qué elemnto están actuando(foco, medio de transmisión o trabajador)}
        \\
        \begin{enumerate}[label=(\alph*)]
          \item \textit{Usar barandillas y redes frente a caídas en una obra.}
            \begin{itemize}
              \item 
            \end{itemize}
          \item \textit{Engrasar una máquina para reducir su ruido}
            \begin{itemize}
              \item 
            \end{itemize}
          \item \textit{Sustituir el uso de tejas de uralita que contiene amianto por otros productos no cancerígenos}
            \begin{itemize}
              \item 
            \end{itemize}
          \item \textit{Entregar ropa especial contro el frío en cámaras frigoríficas}
            \begin{itemize}
              \item 
            \end{itemize}
          \item \textit{Disponer de una bandeja debajo de los envases de productos químicos almacenados, para recoger posibles derrames.}
            \begin{itemize}
              \item 
            \end{itemize}
          \item \textit{El resguardo de una guillotina de hojas, que debe usarse para que funcione la máquina.}
            \begin{itemize}
              \item
            \end{itemize}
        \end{enumerate}

      \newpage
        \subsection{Ejercicio 8}
          \textbf{8º} \textit{ Belén trabaja en una empresa de cerámica donde se produce polvo en suspensión en la fábrica. La empresa establece las siguientes medidas de prevención y protección de riesgos.}
          \\
          \begin{enumerate}[label=(\alph*)]
            \item \textit{Entrega mascarillas a los trabajadores}

            \item \textit{Utiliza aspiradores de polvo en los lugares de trabajo}

            \item \textit{Sustituye el cloro por el oxígeno que es menos perjudicial}
          \end{enumerate}
          
          \begin{enumerate}[label=(\alph*)]
            \item \textit{Indica cuál es una medida de prevención, cuál una medida de protección colectiva y cuál una medida de protección individual.}
              \begin{itemize}
                \item 
              \end{itemize}
            \item \textit{Indica el orden de utilización de estas 3 medidas}
              \begin{itemize}
                \item 
              \end{itemize}
          \end{enumerate}
  
        \newpage
        \subsection{Ejercicio 9}
          \textbf{9º}  \textit{Indica en las siguientes actuaciones a qué técnica preventiva hacen referencia(seguridad, higiene industrial,ergonomía,psicosociología o medicina preventiva)}
          \\
          \begin{enumerate}[label=(\alph*)]
            \item \textit{Medir el aire de una fábrica para detectar la presencia de cvontaminantes químicos.}
              \begin{itemize}
                \item 
              \end{itemize}
            \item \textit{Diseñar herramienteas seguras que impidan el contacto eléctrico}
              \begin{itemize}
                \item 
              \end{itemize}
            \item \textit{Hacer pausas cada 2 horas de 10 minutos que disminuya la fatiga frente al ordenador.}
              \begin{itemize}
                \item 
              \end{itemize}
            \item \textit{Realizar reconocimientos médicos anuales}
              \begin{itemize}
                \item 
              \end{itemize}
            \item \textit{Elegir sillas cómodas para los trabajadores}
              \begin{itemize}
                \item 
              \end{itemize}
            \item \textit{Organizar el trabajo a través de rotación de tareas para evitar la monotonía.}
              \begin{itemize}
                \item
              \end{itemize}
          \end{enumerate}
  \newpage
    \section{TEMA 2}
      \subsection{Ejercicio 1}
        \textbf{1º} \textit{Sandra va a abrir un salón de belleza y contratar a algunas trabajadoras. Le informan que debe elaborar los documentos básicos de prevención de riesgos, por lo que decide consultar a su 
        gestoría cuales son y en qué consisten.)}
        \\
        \begin{enumerate}[label=(\alph*)]
          \item \textit{¿Cómo se llama el documento básico sobre prevención que debe integrarse dentro de la organización?}
            \begin{itemize}
              \item 
            \end{itemize}
          \item \textit{¿En qué casos hay que evaluar los riesgos laborales?}
            \begin{itemize}
              \item 
            \end{itemize}
          \item \textit{¿Cuántas evaluaciones de riesgos hay que realizar?}
            \begin{itemize}
              \item 
            \end{itemize}
          \item \textit{Le comentan que debe evaluar cada zona de trabajo,¿es correcto?¿Por qué?}
            \begin{itemize}
              \item 
            \end{itemize}
          \item \textit{¿Qué medidas deben incluirse en el polan de emergencia?}
            \begin{itemize}
              \item 
            \end{itemize}
          \item \textit{¿Puede la empresa elegir por sorteo al personal encargado de adoptar medidas de emergencia?}
            \begin{itemize}
              \item
            \end{itemize}
        \end{enumerate}
    \newpage
      \subsection{Ejercicio 4}
        \textbf{4º} \textit{A un trabajador le entregan el manual de precención de riesgos con los riesgos generales y las medidas de prevención, para que lo firme al ser contratado. Al mismo tiempo, le dicen que se 
        traiga las mascarillas y guantes de casa, y que si no tiene lo puede comprar en cualquier tienda.)}
        \\
        \begin{enumerate}[label=(\alph*)]
          \item \textit{¿Se ha informado bien el trabajador?¿Por qué?}
            \begin{itemize}
              \item 
            \end{itemize}
          \item \textit{Además de informar, ¿qué otra obligación tiene la empresa en el momento de contratar?}
            \begin{itemize}
              \item 
            \end{itemize}
          \item \textit{¿Debe el trabajador comprar los EPIs?¿Por qué?}
            \begin{itemize}
              \item 
            \end{itemize}
        \end{enumerate}
    \newpage
      \subsection{Ejercicio 5}
        \textbf{5º} \textit{Luis Alfonso va a ser contratado durante 1 año. La empresa le pregunta si sabe algo de prevención de riesgos y Luis Alfonso contesta que en clase de FOL algo vio. La empresa le dice que repase el 
        libro de FOL pues debe estar formado en materia de precención de riesgos laborales para poder trabajar, aunque de todas maneras le entregarán el manual de prevención para que también se lo lea.)}
        \\
        \begin{enumerate}[label=(\alph*)]
          \item \textit{¿Sustituye el libro de FOL o el manual de prevención en cuánto a la formación?¿Por qué?}
            \begin{itemize}
              \item 
            \end{itemize}
          \item \textit{¿Existe motivo para dar formación al trabajador?}
            \begin{itemize}
              \item 
            \end{itemize}
          \item \textit{¿Estaría obligado a participar en la formacion del trabajador?}
            \begin{itemize}
              \item 
            \end{itemize}
          \item \textit{¿Sobre que riesgos debe ser la formación?}
            \begin{itemize}
              \item 
            \end{itemize}
          \item \textit{¿En qué horario debe realizarse la formación?}
            \begin{itemize}
              \item 
            \end{itemize}
          \item \textit{Si se realiza fuera de horario, ¿computa como de trabajo?}
            \begin{itemize}
              \item
            \end{itemize}
        \end{enumerate}
      \newpage
        \subsection{Ejercicio 7}
        \textbf{7º} \textit{A José Ramón le indican que antes de empezar a trabjar debe ir a su médico de la seguridad social para pedirle un reconocimiento médico completo, y traer los resultados a la empresa para ver si es 
        apto o no para el trabajo en el taller.)}
        \\
        \begin{enumerate}[label=(\alph*)]
          \item \textit{¿Debe ir el trabajador a su médico de la seguridad social?}
            \begin{itemize}
              \item 
            \end{itemize}
          \item \textit{¿Son correctas las pruebas que le han pedido?¿Por qué?}
            \begin{itemize}
              \item 
            \end{itemize}
          \item \textit{¿Debe recoger él mismo los resultados y entregarlos a la empresa?}
            \begin{itemize}
              \item 
            \end{itemize}
          \item \textit{¿Qué resultados puede tener la empresa de las pruebas?}
            \begin{itemize}
              \item 
            \end{itemize}
          \item \textit{¿Qué resultados puede tener el trabajador de las pruebas?}
            \begin{itemize}
              \item 
            \end{itemize}
          \item \textit{¿Se puede negar José Ramón a realizar el reconocimiento médico?}
            \begin{itemize}
              \item
            \end{itemize}
        \end{enumerate}
      \newpage
        \subsection{Ejercicio 8}
        \textbf{8º} \textit{Una operaria de un almacén está embarazada de 5 meses, por lo que le indica a la dirección que ya no puede manipular las cajas con el peso que conlleva, solicitando que se tomen las medidas 
        oportunas. Además está cansada de tener que trabajar en turno de noche una de cada tres semanas.)}
        \\
        \begin{enumerate}[label=(\alph*)]
          \item \textit{Qué medida debería tomar la empresa en primer lugar?}
            \begin{itemize}
              \item 
            \end{itemize}
          \item \textit{¿Qué medida debería tomar la empresa en segundo lugar?}
            \begin{itemize}
              \item 
            \end{itemize}
          \item \textit{¿Qué medida debería tomar la empresa en tercer lugar?}
            \begin{itemize}
              \item 
            \end{itemize}
          \item \textit{¿Puede trabajar en turno de noche?}
            \begin{itemize}
              \item 
            \end{itemize}
          \item \textit{¿Y en turnos de mañanas y tardes?}
            \begin{itemize}
              \item 
            \end{itemize}
          \item \textit{Si al final se suspendiese el contrato, ¿cuánto cobraría la trabajadora y quién le pagaría?}
            \begin{itemize}
              \item
            \end{itemize}
        \end{enumerate}
      

\end{document}
