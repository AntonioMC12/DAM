\documentclass{article}

\usepackage{wrapfig}
\usepackage{lmodern}
\usepackage[T1]{fontenc}
\usepackage[spanish]{babel}
\usepackage{mathtools}
\usepackage{graphicx}
\usepackage[utf8]{inputenc}
\usepackage{fancyhdr}
\usepackage{enumitem}
\usepackage{hyperref}
  \hypersetup{
      colorlinks=true, %set true if you want colored links
      linktoc=all,     %set to all if you want both sections and subsections linked
      linkcolor=blue,  %choose some color if you want links to stand out
  }



\title{Ejercicios Tema 1 y 2 \\ \textbf{FOL}}
\author{Antonio Muñoz Cubero}
\date{3 de Noviembre de 2020} 
\pagestyle{fancy}

\begin{document}
  \maketitle
    \tableofcontents
      \pagenumbering{roman}
      \lhead[FOL]{FOL}
      \lfoot[IES Francisco De Los Rios]{IES Francisco De Los Rios}
        

  \newpage
    \section{TEMA 1}
      \subsection{Ejercicio 1}
        \textbf{1º} \textit{Indica para el siguiente trabajador de un almacén de cartonaje a qué factores de riesgo está sometido, así como dentro de cada grupo el tipo de factor de riesgo que es (lugares, equipos,físicos,etc).}
        \\
        \begin{enumerate}[label=(\alph*)]
          \item \textit{Debe transportar los palés de cartón en la zenwich (maquinaria de transporte de palés).}
            \begin{itemize}
              \item Está carente de condiciones de seguridad con los equipos de trabajo.
            \end{itemize}
          \item \textit{Mover cajas de 25kg para colocarlas correctamente en las estanterias.}
            \begin{itemize}
              \item Está carente de condiciones ergonómicas por la carga física.
            \end{itemize}
          \item \textit{Tiene un ruido de fondo de 85 decibelios.}
            \begin{itemize}
              \item Está carente de condiciones medioambientales debido a los agentes físicos.
            \end{itemize}
          \item \textit{Utiliza una serradora de cartón para cortar el cartón sobrante en trozos mas pequeños.}
            \begin{itemize}
              \item Está carente de condiciones de seguridad por el equipo de trabajo que usa.
            \end{itemize}
          \item \textit{En verano en el almacén se está al menos a 30ºC.}
            \begin{itemize}
              \item Está carente de condiciones medioambientales debido a los agentes físicos.
            \end{itemize}
          \item \textit{Su trabajo se le hace monótono y aburrido.}
            \begin{itemize}
              \item Está carente de condiciones psicosociales debido a la organización del trabajo.
            \end{itemize}
          \item \textit{Maneja mucha información de los pedidos y las cajas.}
            \begin{itemize}
              \item Está carente de condiciones ergonómicas debido a la carga mental de su trabajo.
            \end{itemize}
          \item \textit{El polvo que levanta el cartón le provoca estornudos y alergias.}
            \begin{itemize}
              \item Está carente de condiciones medioambientales por agentes químicos.
            \end{itemize}
          \item \textit{Pero él es un perfeccionista y podrá con todo sin importarle los riesgos que existan.}
            \begin{itemize}
              \item Está carente de condiciones psicosociales por las características personales.
            \end{itemize}
        \end{enumerate}
    \newpage
      \subsection{Ejercicio 2}
        \textbf{2º} \textit{En una empresa se han dado distintos accidentes a lo largo del año. Los trabajadores quisieran saber si desde el 
        punto de vista legal se considera o no accidente de trabajo para estar de baja laboral. Indica si es accidente de trabajo desde el 
        punto de vista de la Seguridad Social y porqué.}
        \\
        \begin{enumerate}[label=(\alph*)]
          \item \textit{una trabajadora sufre un corte en un dedo al manipular una máquina en el trabajo.}
            \begin{itemize}
              \item Sí es accidente de trabajo, existe lesión física durante su jornada laboral en su puesto de trabajo.
            \end{itemize}
          \item \textit{Un golpe con el coche al dirigirse al trabajo que obliga a ponerse el collarín y reposo}
            \begin{itemize}
              \item Sí es accidente de trabajo, existe lesión física y se considera accidente in itinere.
            \end{itemize}
          \item \textit{El mismo accidente de coche anterior pero ocurrido un domingo por la noche cuando se dirige al domicilio laboral para empezar a trabajar el lunes.}
            \begin{itemize}
              \item Sí es accidente de trabajo, existe lesión física y se considera accidente in itinere.
            \end{itemize}
          \item \textit{Al conserje le indican que acuda al almacén a ayudar en la descarga de un pedido, sufriendo un golpe por la caída de una caja voluminosa.}
            \begin{itemize}
              \item Sí es accidente de trabajo, pues existe lesión aunque estaba haciendo tareas distintas, es reconocido por la seguridad social.
            \end{itemize}
          \item \textit{Una depresión producida por acoso laboral, la cual se demuestra en el juicio que está vinculada al trabajo.}
            \begin{itemize}
              \item Sí es accidente de trabajo, pues se considera lesión psíquica.
            \end{itemize}
          \item \textit{Mientras manipulaba el curvado de acero, un trabajador se distrae hablando con un compañero y se amputa un dedo}
            \begin{itemize}
              \item Puede ser considerado accidente de trabajo pues es imprudencia profesional.
            \end{itemize}
          \item \textit{En una empresa de Alzira se produce una inundación un lunes por la mañana por las lluvias torrenciales, fruto de lo cual un trabjador resbala golpeándose gravemente la cadera.}
            \begin{itemize}
              \item Es considerado accidente de trabajo, porque existe lesion física.
            \end{itemize}
          \item \textit{La directora financiera acude al banco a firmar unos pagarés y durante el camino es atropellada por una moto.}
            \begin{itemize}
              \item Sí es accidente de trabajo, pues está haciendo una tarea distinta pero es recogida dentro del hámbito que permite la seguridad social.
            \end{itemize}
          \item \textit{Un albañil experimentado sufre una caída con consecuencias graves, pues decide que para 4 metros de altura, no se pone arnés.}
            \begin{itemize}
              \item No es considerado accidente de trabajo, debido a que se trata de una imprudencia temeraria o dolo.
            \end{itemize}
        \end{enumerate}
    
    \newpage
      \subsection{Ejercicio 7}
        \textbf{7º} \textit{Clasifica las siguientes medidas según sean medidas de prevención o de ptrotección. Indica sobre qué elemnto están actuando(foco, medio de transmisión o trabajador)}
        \\
        \begin{enumerate}[label=(\alph*)]
          \item \textit{Usar barandillas y redes frente a caídas en una obra.}
            \begin{itemize}
              \item Medidas de protección que actúan en el medio de transmisión
            \end{itemize}
          \item \textit{Engrasar una máquina para reducir su ruido}
            \begin{itemize}
              \item Medidas de prevencion que actúan en el medio de transmisión
            \end{itemize}
          \item \textit{Sustituir el uso de tejas de uralita que contiene amianto por otros productos no cancerígenos}
            \begin{itemize}
              \item Medidas de prevención que actúan en el medio de transmisión
            \end{itemize}
          \item \textit{Entregar ropa especial contro el frío en cámaras frigoríficas}
            \begin{itemize}
              \item Medida de protección individual que actua en el trabajador.
            \end{itemize}
          \item \textit{Disponer de una bandeja debajo de los envases de productos químicos almacenados, para recoger posibles derrames.}
            \begin{itemize}
              \item Medidas de prevención en el foco.
            \end{itemize}
          \item \textit{El resguardo de una guillotina de hojas, que debe usarse para que funcione la máquina.}
            \begin{itemize}
              \item Medidas de protección colectiva.
            \end{itemize}
        \end{enumerate}

      \newpage
        \subsection{Ejercicio 8}
          \textbf{8º} \textit{ Belén trabaja en una empresa de cerámica donde se produce polvo en suspensión en la fábrica. La empresa establece las siguientes medidas de prevención y protección de riesgos.}
          \\
          \begin{enumerate}[label=(\alph*)]
            \item \textit{Entrega mascarillas a los trabajadores}

            \item \textit{Utiliza aspiradores de polvo en los lugares de trabajo}

            \item \textit{Sustituye el cloro por el oxígeno que es menos perjudicial}
          \end{enumerate}
          
          \begin{enumerate}[label=(\alph*)]
            \item \textit{Indica cuál es una medida de prevención, cuál una medida de protección colectiva y cuál una medida de protección individual.}
              \begin{itemize}
                \item La entrega de mascarillas es una medida de protección individual.
                \item Utilizar aspiradores es una medida de protección colectiva.
                \item Sustituir el cloro es una medida de prevención.
              \end{itemize}
            \item \textit{Indica el orden de utilización de estas 3 medidas}
              \begin{itemize}
                \item Primero sustituir el cloro, después poner las aspiradoras y las mascarillas por último.
              \end{itemize}
          \end{enumerate}
  
        \newpage
        \subsection{Ejercicio 9}
          \textbf{9º}  \textit{Indica en las siguientes actuaciones a qué técnica preventiva hacen referencia(seguridad, higiene industrial,ergonomía,psicosociología o medicina preventiva)}
          \\
          \begin{enumerate}[label=(\alph*)]
            \item \textit{Medir el aire de una fábrica para detectar la presencia de cvontaminantes químicos.}
              \begin{itemize}
                \item Higiene industrial.
              \end{itemize}
            \item \textit{Diseñar herramienteas seguras que impidan el contacto eléctrico}
              \begin{itemize}
                \item Seguridad.
              \end{itemize}
            \item \textit{Hacer pausas cada 2 horas de 10 minutos que disminuya la fatiga frente al ordenador.}
              \begin{itemize}
                \item Ergonomía.
              \end{itemize}
            \item \textit{Realizar reconocimientos médicos anuales}
              \begin{itemize}
                \item Medicina del trabajo.
              \end{itemize}
            \item \textit{Elegir sillas cómodas para los trabajadores}
              \begin{itemize}
                \item Ergonomía.
              \end{itemize}
            \item \textit{Organizar el trabajo a través de rotación de tareas para evitar la monotonía.}
              \begin{itemize}
                \item Psicosociología.
              \end{itemize}
          \end{enumerate}
  \newpage
    \section{TEMA 2}
      \subsection{Ejercicio 1}
        \textbf{1º} \textit{Sandra va a abrir un salón de belleza y contratar a algunas trabajadoras. Le informan que debe elaborar los documentos básicos de prevención de riesgos, por lo que decide consultar a su 
        gestoría cuales son y en qué consisten.)}
        \\
        \begin{enumerate}[label=(\alph*)]
          \item \textit{¿Cómo se llama el documento básico sobre prevención que debe integrarse dentro de la organización?}
            \begin{itemize}
              \item Plan de prevención de riesgos laborales.
            \end{itemize}
          \item \textit{¿En qué casos hay que evaluar los riesgos laborales?}
            \begin{itemize}
              \item Cuando no se puedan evitar los riesgos.
            \end{itemize}
          \item \textit{¿Cuántas evaluaciones de riesgos hay que realizar?}
            \begin{itemize}
              \item Debe existir una evaluación inicial y posteriormente otras periódicas cada vez que se produzca un cambio en las condiciones de trabajo o aparezcan daños o inidicios de que las medidas tomadas no son suficientes.
            \end{itemize}
          \item \textit{Le comentan que debe evaluar cada zona de trabajo,¿es correcto?¿Por qué?}
            \begin{itemize}
              \item Sí.
            \end{itemize}
          \item \textit{¿Qué medidas deben incluirse en el polan de emergencia?}
            \begin{itemize}
              \item Las posibles situaciones de emergencia y las medidas necesarias en primeros auxilios, lucha contra incendios y medidas de evacuación.
            \end{itemize}
          \item \textit{¿Puede la empresa elegir por sorteo al personal encargado de adoptar medidas de emergencia?}
            \begin{itemize}
              \item No, el personal debe poseer la formación y los medios necesarios.
            \end{itemize}
        \end{enumerate}
    \newpage
      \subsection{Ejercicio 4}
        \textbf{4º} \textit{A un trabajador le entregan el manual de precención de riesgos con los riesgos generales y las medidas de prevención, para que lo firme al ser contratado. Al mismo tiempo, le dicen que se 
        traiga las mascarillas y guantes de casa, y que si no tiene lo puede comprar en cualquier tienda.)}
        \\
        \begin{enumerate}[label=(\alph*)]
          \item \textit{¿Se ha informado bien el trabajador?¿Por qué?}
            \begin{itemize}
              \item No, los EPIs los debe proporcionar el empresario de forma gratuita.
            \end{itemize}
          \item \textit{Además de informar, ¿qué otra obligación tiene la empresa en el momento de contratar?}
            \begin{itemize}
              \item Debe ofrecer información teórica y práctica que sea suficiente y adaptada a los riesgos del puesto de trabajo.
            \end{itemize}
          \item \textit{¿Debe el trabajador comprar los EPIs?¿Por qué?}
            \begin{itemize}
              \item No, los EPIs los debe proporcionar el empresario de forma gratuita.
            \end{itemize}
        \end{enumerate}
    \newpage
      \subsection{Ejercicio 5}
        \textbf{5º} \textit{Luis Alfonso va a ser contratado durante 1 año. La empresa le pregunta si sabe algo de prevención de riesgos y Luis Alfonso contesta que en clase de FOL algo vio. La empresa le dice que repase el 
        libro de FOL pues debe estar formado en materia de precención de riesgos laborales para poder trabajar, aunque de todas maneras le entregarán el manual de prevención para que también se lo lea.)}
        \\
        \begin{enumerate}[label=(\alph*)]
          \item \textit{¿Sustituye el libro de FOL o el manual de prevención en cuánto a la formación?¿Por qué?}
            \begin{itemize}
              \item No, la formacion debe estar adaptada a los riesgos del puesto de trabajo.
            \end{itemize}
          \item \textit{¿Existe motivo para dar formación al trabajador?}
            \begin{itemize}
              \item Sí.
            \end{itemize}
          \item \textit{¿Estaría obligado a participar en la formacion del trabajador?}
            \begin{itemize}
              \item Sí.
            \end{itemize}
          \item \textit{¿Sobre que riesgos debe ser la formación?}
            \begin{itemize}
              \item Sobre los riesgos de su puesto de trabajo,
            \end{itemize}
          \item \textit{¿En qué horario debe realizarse la formación?}
            \begin{itemize}
              \item Si es posible en horario de trabajo, si no, este se descuenta del tiempo de trabajo.
            \end{itemize}
          \item \textit{Si se realiza fuera de horario, ¿computa como de trabajo?}
            \begin{itemize}
              \item Sí.
            \end{itemize}
        \end{enumerate}
      \newpage
        \subsection{Ejercicio 7}
        \textbf{7º} \textit{A José Ramón le indican que antes de empezar a trabjar debe ir a su médico de la seguridad social para pedirle un reconocimiento médico completo, y traer los resultados a la empresa para ver si es 
        apto o no para el trabajo en el taller.)}
        \\
        \begin{enumerate}[label=(\alph*)]
          \item \textit{¿Debe ir el trabajador a su médico de la seguridad social?}
            \begin{itemize}
              \item No, debe ser a través de los servicios de prevención.
            \end{itemize}
          \item \textit{¿Son correctas las pruebas que le han pedido?¿Por qué?}
            \begin{itemize}
              \item No son correctas, solo se deben realizar las pruebas imprescindibles y necesarias.
            \end{itemize}
          \item \textit{¿Debe recoger él mismo los resultados y entregarlos a la empresa?}
            \begin{itemize}
              \item No debe entregarlos a la empresa.
            \end{itemize}
          \item \textit{¿Qué resultados puede tener la empresa de las pruebas?}
            \begin{itemize}
              \item Si es o no apto para el puesto de trabajo.
            \end{itemize}
          \item \textit{¿Qué resultados puede tener el trabajador de las pruebas?}
            \begin{itemize}
              \item Los resultados completos.
            \end{itemize}
          \item \textit{¿Se puede negar José Ramón a realizar el reconocimiento médico?}
            \begin{itemize}
              \item Sí, a menos que sea obligatorio por ley.
            \end{itemize}
        \end{enumerate}
      \newpage
        \subsection{Ejercicio 8}
        \textbf{8º} \textit{Una operaria de un almacén está embarazada de 5 meses, por lo que le indica a la dirección que ya no puede manipular las cajas con el peso que conlleva, solicitando que se tomen las medidas 
        oportunas. Además está cansada de tener que trabajar en turno de noche una de cada tres semanas.)}
        \\
        \begin{enumerate}[label=(\alph*)]
          \item \textit{Qué medida debería tomar la empresa en primer lugar?}
            \begin{itemize}
              \item Una evalu específica de los riesgos a los que está sometida la trabajadora y determina que puestos son los que están exentos de riestgos.
            \end{itemize}
          \item \textit{¿Qué medida debería tomar la empresa en segundo lugar?}
            \begin{itemize}
              \item Adaptar las condiciones de trabajo.
            \end{itemize}
          \item \textit{¿Qué medida debería tomar la empresa en tercer lugar?}
            \begin{itemize}
              \item Cambiar a la trabajadora a un pueesto compatible.
            \end{itemize}
          \item \textit{¿Puede trabajar en turno de noche?}
            \begin{itemize}
              \item No.
            \end{itemize}
          \item \textit{¿Y en turnos de mañanas y tardes?}
            \begin{itemize}
              \item Sí.
            \end{itemize}
          \item \textit{Si al final se suspendiese el contrato, ¿cuánto cobraría la trabajadora y quién le pagaría?}
            \begin{itemize}
              \item La seguridad social.
            \end{itemize}
        \end{enumerate}
      

\end{document}
