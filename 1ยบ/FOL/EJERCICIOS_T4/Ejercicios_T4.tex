\documentclass{article}

\usepackage{wrapfig}
\usepackage{lmodern}
\usepackage[T1]{fontenc}
\usepackage[spanish]{babel}
\usepackage{mathtools}
\usepackage{graphicx}
\usepackage[utf8]{inputenc}
\usepackage{fancyhdr}
\usepackage{enumitem}
\usepackage{hyperref}
  \hypersetup{
      colorlinks=true, %set true if you want colored links
      linktoc=all,     %set to all if you want both sections and subsections linked
      linkcolor=blue,  %choose some color if you want links to stand out
  }



\title{Ejercicios Tema 3 \\ \textbf{FOL}}
\author{Antonio Muñoz Cubero}
\date{22 de Noviembre de 2020} 
\pagestyle{fancy}

\begin{document}
  \maketitle
    \tableofcontents
      \pagenumbering{arabic}
      \lhead[FOL]{FOL}
      \lfoot[IES Francisco De Los Rios]{IES Francisco De Los Rios}
        

  \newpage
    \section{TEMA 4}
      \subsection{Ejercicio 1}
        \begin{enumerate}[label=(\alph*)]
          \item \textit{Utilizar los ascensores y salir corriendo lo máximo posible:}
            \begin{itemize}
              \item No son adecuados. Porque si el hueco del ascensor se llena de humo, en caso de incendio el hueco del ascensor funciona como chimenea, y no correr para que no cunda el pánico y salir mas ordenadamente sin dejar a nadie atrás.
            \end{itemize}
          \item \textit{Volver a los despachos a por los abrigos y a cerrar con llave:}
            \begin{itemize}
              \item No es adecuado. Porque es peligroso y puedes quedarte atrapado.
            \end{itemize}
          \item \textit{Si hay humo cubrirse la nariz y la boca con un pañuelo mojado y andar a gatas:}
            \begin{itemize}
              \item Si es adecuado. Porque se puede respirar mejor en caso de humo a través de un trapo mojado y a gatas porque el aire caliente sube.
            \end{itemize}
          \item \textit{Dirigirse al punto de reunión de forma ordenada según las instrucciones de los equipos de alarma y evacuación:}
            \begin{itemize}
              \item Si es adecuado. Porque así se tiene un orden y se puede ver si falta alguien.
            \end{itemize}
          \item \textit{Si se prende fuego e nla ropa no correr sino rodar por el suelo:}
            \begin{itemize}
              \item Si es adecuado. Porque puedes apagar el fuego de tu ropa rodando.
            \end{itemize}
          \item \textit{Volver a por el móvil y dejar las puertas abiertas:}
            \begin{itemize}
              \item No es adecuado. Porque es peligroso volver al edificio y además hay que dejar las puertas cerradas para intentar contener el fuego.
            \end{itemize}
        \end{enumerate}

      \subsection{Ejercicio 4}
        \begin{enumerate}[label=(\alph*)]
          \item \textit{Un chico con un corte en la mano}
            \begin{itemize}
              \item Tarjeta verde.
            \end{itemize}
          \item \textit{Una señora mayor que está incosciente}
            \begin{itemize}
              \item Tarjeta roja.
            \end{itemize}
          \item \textit{Un señor con una fractura dolorosa en la pierna}
            \begin{itemize}
              \item Tarjeta amarilla.
            \end{itemize}
          \item \textit{Una chica que no para de sangrar por el brazo}
            \begin{itemize}
              \item Tarjeta roja.
            \end{itemize}
          \item \textit{Una persona que sangra por los oidos}
            \begin{itemize}
              \item Tarjeta roja.
            \end{itemize}
          \item \textit{Un niño que tiene sangre en la rodilla del golpe}
            \begin{itemize}
              \item Tarjeta Verde.
            \end{itemize}
        \end{enumerate}

      \subsection{Ejercicio 6}
        \begin{enumerate}[label=(\alph*)]
          \item \textit{¿Qué debe hacer lo primero con la persona que no responde y está incosciente?}
            \begin{itemize}
              \item Pedimos ayuda urgente y pasamos a abrir la vía respiratoria.
            \end{itemize}
          \item \textit{¿Qué debe hacerse con las personas que sí responden?}
            \begin{itemize}
              \item Se atienden otras posibles lesiones como hemorragias, heridas, fracturas, etc.
            \end{itemize}
          \item \textit{A la persona incosciente, le abre la vía respiratoria y ve que está obstruida por la lengua,¿Qué maniobra haría para abrir la vía?}
            \begin{itemize}
              \item Debe aplicarse la técnica de la \textbf{hiperextensión} del cuello.
            \end{itemize}
          \item \textit{Si no está segura si esa persona respira, ¿Cómo puede averigarlo?¿En cuanto tiempo?}
            \begin{itemize}
              \item Aplicando la técnica del "ver,oír,sentir" pero de forma rápida, no pudiendo durar mas de 10 segundos.
            \end{itemize}
          \item \textit{¿En qué caso colocará la a la persona incosciente en PLS?}
            \begin{itemize}
              \item Si la persona respira
            \end{itemize}
          \item \textit{Si a pesar de abrir las vias no respirase, ¿qué hará después de llamar al 112?}
            \begin{itemize}
              \item Nosotros comprobamos el pulso e iniciamos el masaje cardiaco.
            \end{itemize}
        \end{enumerate}

    \newpage
      \subsection{Ejercicio 7}
        \begin{enumerate}[label=(\alph*)]
          \item \textit{¿Qué debería hacer a continuación el socorrista?}
            \begin{itemize}
              \item Llamar al 112, comprobar el pulso e iniciar el masaje cardiaco.
            \end{itemize}
          \item \textit{¿Dónde se comprueba si tiene pulso?}
            \begin{itemize}
              \item En el cuello
            \end{itemize}
          \item \textit{Si no tiene pulso, ¿qué debe hacer a continuación?}
            \begin{itemize}
              \item Iniciar el masaje cardiaco.
            \end{itemize}
          \item \textit{Si termina aplicando el masaje cardiaco, ¿A que ritmo debe ir?}
            \begin{itemize}
              \item 100 repeticiones por minuto sin superar las 120 repeticiones. Siempre se dice que hay que dar el masaje al ritmo de "La Macarena".
            \end{itemize}
          \item \textit{¿Dónde está el punto donde debe aplicarse el masaje?}
            \begin{itemize}
              \item En la parte inferior del esternón.
            \end{itemize}
          \item \textit{¿A qué ritmo se aplica el boca a boca junto a las compresiones?}
            \begin{itemize}
              \item 30 compresiones y 2 ventilaciones boca a boca.
            \end{itemize}
        \end{enumerate}

      \subsection{Ejercicio 11}
        \begin{enumerate}[label=(\alph*)]
          \item \textit{¿Qué tipo de agente ha provocado el fuego en ambos casos?}
            \begin{itemize}
              \item Un agente eléctrico.
            \end{itemize}
          \item \textit{¿Puede apargse el fuego de gasolina y alcohol con agua?¿Por qué?}
            \begin{itemize}
              \item No, ya que lo extendería.
            \end{itemize}
          \item \textit{¿Puede apagarse el fuego de salfumán con agua?¿Por qué?}
            \begin{itemize}
              \item Si, pero con contidades de agua muy abundantes.
            \end{itemize}
          \item \textit{¿Cómo se debe apagar el fuego con gasolina y alcohol en el brazo de Luis?}
            \begin{itemize}
              \item Con una manta, ropa no sintetica o rodando en el suelo.
            \end{itemize}
          \item \textit{¿Durante cuánto tiempo hay que aplicar agua en el caso de lejía de Enrique?}
            \begin{itemize}
              \item Entre 20 y 30 minutos.
            \end{itemize}
          \item \textit{¿Se puede aplicar el extintor sobre una persona en algún caso?}
            \begin{itemize}
              \item En el caso de los líquidos inflamables siempre que sea de polvo ABC o espuma fisica.
            \end{itemize}
        \end{enumerate}

      \subsection{Ejercicio 13}
        \begin{enumerate}[label=(\alph*)]
          \item \textit{¿Es correcto comenzar con el abrazo del oso en ambos casos?}
            \begin{itemize}
              \item No, solo se efectuá en caso de que no tosa.
            \end{itemize}
          \item \textit{En el caso de Bartolo que tose, ¿cómo hay que actuar en su caso?}
            \begin{itemize}
              \item Animarle a seguir tosiendo y no darle golpes en la espalda
            \end{itemize}
          \item \textit{En el caso de Raquel que no puede hablar, ¿qué hay que hacer en primer lugar?}
            \begin{itemize}
              \item Se le inclina un poco hacia adelante y se le dan palmadas en la espalda.
            \end{itemize}
          \item \textit{¿Cuántas palmadas hay que dar seguidas antes del abrazo del oso?}
            \begin{itemize}
              \item 5
            \end{itemize}
          \item \textit{Si Raquel cayera incosciente, ¿se le levanta para aplicarle el abrazo del oso?¿Cómo hay que actuar en ese caso?}
            \begin{itemize}
              \item Con el paciente tumbado boca arriba aplicar compresiones al igual que la RCP y retirar elobjeto con los dedos en forma de pinza una vez llegue a la boca.
            \end{itemize}
        \end{enumerate}
\end{document}